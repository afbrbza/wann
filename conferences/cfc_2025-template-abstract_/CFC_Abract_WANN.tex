\documentclass{cfc2025}
\usepackage{pslatex}
%\usepackage{amsmath}
%\usepackage{amsfonts}
%\usepackage{amssymb}
\thispagestyle{empty}

\title{Development of Artificial Neural Networks for Predicting Production Rates in Horizontal Wellbores}

\author{Nathan Shauer$^{*}$, Thiago Dias dos Santos$^{*}$, Giovane Avancini$^{*}$ and Philippe R. B. Devloo$^{*}$}

\address{$^{*}$ School of Civil Engineering, Architecture and Urban Planning\\
Universidade Estadual de Campinas (Unicamp)\\
Campinas, SP, Brazil\\
e-mail: shauer@unicamp.br
}

\begin{document}

\thispagestyle{empty}

This research focuses on the development and application of surrogate models for predicting production rates in horizontal wellbores through the utilization of Artificial Neural Networks (ANN). The ANN is trained using data derived from a locally conservative finite element porous media flow code. This code employs a mixed Finite Element Method (FEM) formulation, which facilitates the straightforward sampling of flow and its derivative along the wellbore. The obtained data is then used to estimate a pseudo resistivity for the wellbore. This pseudo resistivity is subsequently incorporated into a coupled system of one-dimensional differential equations that govern the flow within the wellbore. These equations are solved using a Runge-Kutta scheme. The proposed methodology demonstrates the capability to achieve accurate results with considerably less computational time than a standard numerical solver.

\section*{ACKNOWLEDGMENTS}

The authors acknowledge the School of Civil Engineering, Architecture and Urban Planning at Unicamp for its support of this research.

\begin{thebibliography}{99}
\fontsize{11}{12}\selectfont

\bibitem{Santos} Santos, Thiago Dias dos. Aplicação de redes neurais artificiais em simulação numérica do acoplamento poço-reservatório. 2012. Dissertation (master's). Universidade Estadual de Campinas. 
\end{thebibliography}


\end{document}


